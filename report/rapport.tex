\documentclass[a4paper,11pt,titlepage]{article}
\usepackage[utf8]{inputenc}
\usepackage[T1]{fontenc}
\usepackage[frenchb]{babel}
\usepackage{lmodern}
\usepackage[final,expansion=true,protrusion=true]{microtype}
\usepackage[top=30mm,bottom=20mm,left=18mm,right=18mm,headheight=20mm]{geometry}
\usepackage[hidelinks,unicode]{hyperref}
\usepackage{graphicx}
\usepackage{fancyhdr}
\usepackage{amsmath}
\usepackage{kantlipsum}

\title{SYM 2015\\--\\Utilisation de données environnementales}
\author{%
Laurent Girod \href{mailto:laurent.girod@heig-vd.ch}{\texttt{<laurent.girod@heig-vd.ch>}}\\%
Nicolas Kobel \href{mailto:nicolas.kobel@heig-vd.ch}{\texttt{<nicolas.kobel@heig-vd.ch>}}%
%Rick Wertenbroek \href{mailto:rick.wertenbroek@heig-vd.ch}{\texttt{<rick.wertenbroek@heig-vd.ch>}}\\%
%Cyrill Zundler \href{mailto:cyrill.zundler@heig-vd.ch}{\texttt{<cyrill.zundler@heig-vd.ch>}}%
}
\date{\today}

% Uncomment if sans serif font is preferred.
%\renewcommand*{\familydefault}{\sfdefault}
\renewcommand{\headrulewidth}{1pt}

\setlength{\parindent}{0pt}
\setlength{\parskip}{2ex}

\lhead{SYM 2015 -- Utilisation de données environnementales}
\chead{}
\rhead{\includegraphics[height=8mm]{logo.pdf}}
\lfoot{}
\cfoot{\thepage}
\rfoot{}

\pagestyle{fancy}

\begin{document}
\maketitle{}
\section{Balises NFC}
\subsection*{Questions}
\textit{Sachant que les collaborateurs de l'entreprise UBIQOMP SA se déplacent en véhiculant des
informations extrêmement précieuses dans leurs dispositifs informatiques mobiles (munis de dispositifs de
lecture NFC), et qu'ils sont amenés à se rendre dans des zones à risque, un expert a fait les estimations
suivantes:}
\begin{itemize}
	\item \textit{La probabilité de vol d'un mobile par une personne malintentionnée et capable d'utiliser les
		données à des fins préjudiciables pour la société est de 1\%}
	\item \textit{La probabilité que le mot de passe puisse être découvert, soit par analyse des traces de doigts sur
		l'écran, soit par observation en cours d'utilisation est de 5\%}
	\item \textit{La probabilité de vol du porte-clés est de 0.1\%}
	\item \textit{Environ 10\% des criminels susceptibles d'accéder aux données du mobile sait que le porte-clés
		donne accès au mobile}
\end{itemize}

\textit{Quelle est la probabilité moyenne globale que des données soient perdues, dans le cas où il faut la
balise ET le mot de passe, ainsi que dans le cas où il faut la balise OU le mot de passe, ou encore le cas où
seule la balise est nécessaire? En d'autres termes, si l'on envoie cent collaborateurs dans la géographie,
quel est le risque encouru de vol de données sensibles? Mettre vos conclusions en rapport avec l'inconfort
subjectif de chaque solution.}

Soient:
\begin{itemize}
	\item $M$ l'évènement ``le mobile est volé par une personne malintentionnée et capable d'utiliser les données
		à des fins préjudiciables pour la société'',
	\item $P$ l'évènement ``le mot de passe est découvert'',
	\item $B$ l'évènement ``la balise (sur le porte-clés) est volée'',
	\item $S$ l'évènement ``le criminel sais que le porte clé donne accès au mobile''.
\end{itemize}

On a les probabilités suivantes:
\begin{align}
	P(M) = 0.01 \\
	P(P) = 0.05 \\
	P(B) = 0.001 \\
	P(S | M) = 0.1 \\
\end{align}

On suppose que toutes ces probabilités sont indépendantes.

La probabilité que des données soient perdues dans le cas où il faut la balise \textbf{et} le mot de passe est:
\begin{multline}
	P(S\cap P\cap B)
	= P(S) P(P) P(B)
	= P(S | M) P(M) P(P) P(B)
	\\
	= 0.1\cdot 0.01\cdot 0.05\cdot 0.001
	= 5.00\cdot 10^{-8}
\end{multline}

La probabilité de perte de données dans ce cas est donc 1 chance sur 20 milions.

La probabilité que des données soient perdues dans le cas où il faut la balise \textbf{ou} le mot de passe est:
\begin{multline}
	P\left((S\cap B)\cup (M\cap P)\right)
	= \left(P(S) P(B)\right) + \left(P(M) P(P)\right)
	= \left(P(S | M) P(M) P(B)\right) + \left(P(M) P(P)\right)
	\\
	= (0.1\cdot 0.01\cdot 0.001) + (0.01\cdot 0.05)
	= 5.01\cdot 10^{-4}
\end{multline}

La probabilité de perte de données dans ce cas est donc un peu plus que 1 chance sur 2000.

La probabilité que des données soient perdues dans le cas où seule le mot de passe est nécessaire est:
\begin{multline}
	P\left(M\cap P\right)
	= \left(P(M) P(P)\right)
	= \left(P(M) P(P)\right)
	\\
	= (0.01\cdot 0.05)
	= 5.00\cdot 10^{-4}
\end{multline}

La probabilité de perte de données dans ce cas est donc 1 chance sur 2000.

La probabilité que des données soient perdues dans le cas où seule la balise est nécessaire est:
\begin{multline}
	P\left(S\cap B\right)
	= \left(P(S) P(B)\right)
	= \left(P(S | M) P(M) P(B)\right)
	\\
	= (0.1\cdot 0.01\cdot 0.001)
	= 1.00\cdot 10^{-6}
\end{multline}

La probabilité de perte de données dans ce cas est donc 1 chance sur un milion.

%Mettre vos conclusions en rapport avec l'inconfort subjectif de chaque solution.

%TODO
\kant[1]

\textit{Peut-on améliorer la situation en introduisant un contrôle des informations d'authentification par un
serveur éloigné (transmission d'un "hash" genre MD5 du mot de passe et de la balise NFC)? Si oui, à quelles
conditions? Quels inconvénients?}


\textit{Proposer une stratégie permettant à la société UBIQOMP SA d'améliorer grandement son bilan
sécuritaire, en détailler les inconvénients pour les utilisateurs et pour la société.}

\section{Codes barres}
\subsection*{Questions}
\textit{Comparer la technologie à codes-barres et la technologie NFC, du point de vue d'une utilisation dans
des applications pour smartphones, dans une optique:}
\begin{itemize}
\item \textit{Professionnelle (Authentification dure, Lettres de créance, accès bases de données, clés de
	chiffrage)}
\item \textit{Grand public (ticketing, access control, e-paiement)}
\item \textit{Ludique (preuves d'achat, identité électronique pour gaming, etc...)}
\item \textit{Financier (coûts impliqués par le déploiement de la technologie, possibilités de recyclage, etc...)}
\end{itemize}

\textit{Est-il possible sans autres de remplacer la puce NFC par une carte à code QR dans l'exercice
précédent, moyennant un simple  ajustement du code pour lire le code à barres. (Note : il existe de
nombreux  générateurs  de codes  QR sur le web, comme par exemple
\url{http://www.myfeelback.com/fr/generateur-qr-code-gratuit-personnalisable}).}

\textit{L'un des projets que poursuit iict/useraware actuellement est l'identification de patients dans de
grands centres hospitaliers. Reprendre les arguments développés dans le point précédent dans cette
optique particulière, en tenant compte du fait que l'identification du patient est presque toujours couplée à
celle du soignant qui effectue un acte médical envers ce patient.}

\textit{Un autre projet qui a été entrepris en collaboration avec la société sysmosoft sa (et qui a aussi fait
l'objet d'un projet de master à la HEIG-VD) est la reconnaissance  de contexte. Ainsi, selon le contexte dans
lequel on se trouve, on va être plus ou moins tolérant vis-à-vis de l'accès à l'information. Imaginer les
paramètres que l'on peut associer à un contexte, et pour chacun d'eux, estimer le degré de ``fiabilité'' de
ce paramètre dans l'optique d'une méthode heuristique d'identification d'un contexte d'utilisation. Mettre
en rapport les inconvénients et les avantages liés à l'utilisation du paramètre en question.}

\end{document}
